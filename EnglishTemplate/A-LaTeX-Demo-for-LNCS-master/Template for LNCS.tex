% !TeX spellcheck = <none>
% !TeX encoding = GB2312
% This is lnicst.tex the demonstration file of the LaTeX macro package for
% Lecture Notes of the Institute for Computer Sciences, Social-Informatics
% and Telecommunications Engineering series from Springer-Verlag.
% It serves as a template for authors as well.
% version 1.0 for LaTeX2e
%
\documentclass[lnicst]{svmultln}
%
\usepackage{makeidx}  % allows for indexgeneration
% \makeindex          % be prepared for an author index
%
\usepackage{amssymb}
\setcounter{tocdepth}{3}
\usepackage{graphicx}
\graphicspath{{./graphics/}}

\usepackage{url}

\usepackage[pdfpagelabels,hypertexnames=false,breaklinks=true,bookmarksopen=true,bookmarksopenlevel=2]{hyperref}
%%%%%%%%%%%%%%%%%%%%%%%%%%%%%%%
\usepackage{booktabs}
\usepackage{threeparttable}
\usepackage{color}
%%%%%%%%%%%%%%%%%%%%%%%%%%%%%%%

\usepackage{color}

\makeatletter
\newif\if@restonecol
\makeatother
\let\algorithm\relax
\let\endalgorithm\relax
\usepackage[linesnumbered,ruled,vlined]{algorithm2e}%[ruled,vlined]{
\usepackage{algpseudocode}
\usepackage{amsmath}
\renewcommand{\algorithmicrequire}{\textbf{Input:}}  % Use Input in the format of Algorithm
\renewcommand{\algorithmicensure}{\textbf{Output:}} % Use Output in the format of Algorithm


\begin{document}
%
\mainmatter              % start of the contribution
%
\title{this is only for english LNCS}
%

\author{Jinliang Xu\inst{1} \and Jinliang Xu\inst{2}}
%
%\authorrunning{Ivar Ekeland et al.}   % abbreviated author list (for running head)
%
%%%% list of authors for the TOC (use if author list has to be modified)
\tocauthor{Jinliang Xu, Jinliang Xu}
%
\institute{Beijing University of Posts and Telecommunications, China,\\
\email{jlxu@bupt.edu.cn}
\and
Beijing University of Posts and Telecommunications, China,\\
\email{jlxu@bupt.edu.cn}}

\maketitle              % typeset the title of the contribution


\begin{abstract}       
The abstract should summarize the contents of the paper
using at least 70 and at most 150 words. It will be set in 9-point
font size and be inset 1.0 cm from the right and left margins.
There will be two blank lines before and after the Abstract.

\keywords {high-level Petri nets, net components}
\end{abstract}
%
\section{Section Title}
%
%
In this section, we will consider the case when the Hamiltonian $H(x)$
is autonomous. For the sake of simplicity, we shall also assume that it
is $C^{1}$.

We shall first consider the question of nontriviality, within the
general framework of
$\left(A_{\infty},B_{\infty}\right)$-subquadratic Hamiltonians. In
the second subsection, we shall look into the special case when $H$ is
$\left(0,b_{\infty}\right)$-subquadratic,
and we shall try to derive additional information.
%
\subsection{Subsection Title}
%
Theorem~\ref{ghou:pre} tells us that if $\lambda +\gamma < 0$, the
boundary-value problem:
\begin{equation}
\begin{array}{rcl}
  \dot{x}&=&JH' (x)\\
  x(0)&=&x (T)
\end{array}
\end{equation}


%
\begin{proposition}
Assume $H'(0)=0$ and $ H(0)=0$. Set:
\begin{equation}
  \delta := \liminf_{x\to 0} 2 N (x) \left\|x\right\|^{-2}\ .
  \label{eq:one}
\end{equation}

If $\gamma < - \lambda < \delta$,
the solution $\overline{u}$ is non-zero:
\begin{equation}
  \overline{x} (t) \ne 0\ \ \ \forall t\ .
\end{equation}
\end{proposition}
%
\begin{proof}
Condition (\ref{eq:one}) means that, for every
$\delta ' > \delta$, there is some $\varepsilon > 0$ such that
\begin{equation}
  \left\|x\right\| \le \varepsilon \Rightarrow N (x) \le
  \frac{\delta '}{2} \left\|x\right\|^{2}\ .
\end{equation}


\begin{figure}
\centering
\includegraphics[height=6.2cm]{ratings_groups}
\caption{A figure example.}
\label{fig:example}
\end{figure}



On the other hand, we check directly that $\psi (0) = 0$. This shows
that 0 cannot be a minimizer of $\psi$, not even a local one.
So $\overline{u} \ne 0$ and
$\overline{u} \ne \Lambda_{o}^{-1} (0) = 0$. \qed
\end{proof}
%
\begin{corollary}
Assume $H$ is $C^{2}$ and
$\left(a_{\infty},b_{\infty}\right)$-subquadratic at infinity. Let
$\xi_{1},\allowbreak\dots,\allowbreak\xi_{N}$  be the
equilibria, that is, the solutions of $H' (\xi ) = 0$.
\end{corollary}
%

\begin{proof}
The condition $\gamma < -\lambda < \delta$ now becomes:
\begin{equation}
  b_{\infty} - a_{\infty} <
  - \frac{2\pi}{T} k_{o} -a_{\infty} < \omega -a_{\infty}
\end{equation}
which is precisely condition .\qed
\end{proof}
%
%
\paragraph{Notes and Comments.}
The results in this section are a
refined version of \cite{smit:wat};
the minimality result of Proposition
14 was the first of its kind.

To understand the nontriviality conditions, such as the one in formula
(\ref{eq:four}), one may think of a one-parameter family
$x_{T}$, $T\in \left(2\pi\omega^{-1}, 2\pi b_{\infty}^{-1}\right)$
of periodic solutions, $x_{T} (0) = x_{T} (T)$,
with $x_{T}$ going away to infinity when $T\to 2\pi \omega^{-1}$,
which is the period of the linearized system at 0.

\begin{table}
\caption{This is the example table taken out of {\it The
\TeX{}book,} p.\,246}
\begin{center}
\begin{tabular}{r@{\quad}rl}
\hline
\multicolumn{1}{l}{\rule{0pt}{12pt}
                   Year}&\multicolumn{2}{l}{World population}\\[2pt]
\hline\rule{0pt}{12pt}
8000 B.C.  &     5,000,000& \\
  50 A.D.  &   200,000,000& \\
1650 A.D.  &   500,000,000& \\
1945 A.D.  & 2,300,000,000& \\
1980 A.D.  & 4,400,000,000& \\[2pt]
\hline
\end{tabular}
\end{center}
\end{table}
%
\begin{theorem} [Ghoussoub-Preiss]\label{ghou:pre}
Assume $H(t,x)$ is
$(0,\varepsilon )$-subquadratic at
infinity for all $\varepsilon > 0$, and $T$-periodic in $t$
\begin{equation}
  H (t,\cdot )\ \ \ \ \ {\rm is\ convex}\ \ \forall t
\end{equation}
\begin{equation}
  H (\cdot ,x)\ \ \ \ \ {\rm is}\ \ T{\rm -periodic}\ \ \forall x
\end{equation}
\begin{equation}
  H (t,x)\ge n\left(\left\|x\right\|\right)\ \ \ \ \
  {\rm with}\ \ n (s)s^{-1}\to \infty\ \ {\rm as}\ \ s\to \infty
\end{equation}
\begin{equation}
  \forall \varepsilon > 0\ ,\ \ \ \exists c\ :\
  H(t,x) \le \frac{\varepsilon}{2}\left\|x\right\|^{2} + c\ .
\end{equation}

Assume also that $H$ is $C^{2}$, and $H'' (t,x)$ is positive definite
everywhere. Then there is a sequence $x_{k}$, $k\in \bbbn$, of
$kT$-periodic solutions of the system
\begin{equation}
  \dot{x} = JH' (t,x)
\end{equation}
such that, for every $k\in \bbbn$, there is some $p_{o}\in\bbbn$ with:
\begin{equation}
  p\ge p_{o}\Rightarrow x_{pk} \ne x_{k}\ .
\end{equation}
\qed
\end{theorem}
%
%
\begin{definition}
Let $A_{\infty} (t)$ and $B_{\infty} (t)$ be symmetric
operators in $\bbbr^{2n}$, depending continuously on
$t\in [0,T]$, such that
$A_{\infty} (t) \le B_{\infty} (t)$ for all $t$.

Note that, if $k<0$, it is not convex.
\end{definition}
%

The first results on subharmonics were
obtained by Foster and Kesselman in \cite{minson2015crowdsourced,kazai2016quality}, who showed the existence of
infinitely many subharmonics both in the subquadratic and superquadratic
case, with suitable growth conditions on $H'$. Again the duality
approach enabled Foster and Waterman in \cite{konevcny2010early} to treat the
same problem in the convex-subquadratic case, with growth conditions on
$H$ only.

Recently, Smith and Waterman (see \cite{kazai2016quality} and May et al. \cite{konevcny2010early})
have obtained lower bound on the number of subharmonics of period $kT$,
based on symmetry considerations and on pinching estimates, as in
Sect.~5.2 of this article \cite{kazai2016quality}.

%
% ---- Bibliography ----
%
\bibliographystyle{plain}
\bibliography{reference}
%
\end{document}
